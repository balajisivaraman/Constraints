\begin{section}{Conclusion}
\begin{frame}
\frametitle{Conclusion}
\begin{itemize}
\item Constraints make for more readable and refactorable code
\item Always reach for the most constrained abstraction, leaving the implementor with fewer choices
\item Parametricity and Higher Kinded Types enable a lot of reuse
\end{itemize}
\end{frame}
\end{section}
