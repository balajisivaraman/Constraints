\documentclass[bigger]{beamer}
\usepackage[utf8]{inputenc}
\usepackage[T1]{fontenc}
\usepackage{fixltx2e}
\usepackage{graphicx}
\usepackage{grffile}
\usepackage{longtable}
\usepackage{wrapfig}
\usepackage{rotating}
\usepackage[normalem]{ulem}
\usepackage{amsmath}
\usepackage{textcomp}
\usepackage{amssymb}
\usepackage{capt-of}
\usepackage{hyperref}
\usepackage{listings}

% \usetheme{Copenhagen}
\usetheme{metropolis}
\metroset{numbering=fraction}
\setsansfont{Noto Sans}
\setmonofont{PragmataPro}

\author{Balaji Sivaraman (\href{https://twitter.com/balajisivaraman}{@balajisivaraman})}
\date{November 3, 2016}
\title{Free Yourself With Constraints}
\institute{ThoughtWorks}
\hypersetup{
 pdfauthor={Balaji Sivaraman},
 pdftitle={Free Yourself With Constraints},
 pdfkeywords={functional programming, parametricity, typed fp, higher-kinded types},
 pdfsubject={A talk on how constraints help us write readable programs},
 pdflang={English}}

\begin{document}

\xdefinecolor{darkgreen}{rgb}{0,0.35,0}
\lstset{
  tabsize=2,
  basicstyle=\ttfamily,
  moredelim=**[is][\btHL]{`}{`}
}
\lstdefinestyle{java}{
  language=java,
  basicstyle=\footnotesize\ttfamily,
  stringstyle=\color{darkgreen}\ttfamily,
  commentstyle=\color{gray}\ttfamily,
  keywordstyle=\footnotesize\color{blue}\ttfamily,
  tabsize=2,
  moredelim=**[is][\btHL]{`}{`}
}
\lstdefinelanguage{scala}{
  morekeywords={abstract,case,catch,class,def,%
    do,else,extends,false,final,finally,%
    for,forSome,if,implicit,import,lazy,match,%
    new,null,object,override,package,%
    private,protected,requires,return,sealed,%
    super,this,throw,trait,true,try,%
    type,val,var,while,with,yield},
  otherkeywords={=,=>,<-,<\%,<:,>:,\#,@},
  sensitive=true,
  morecomment=[l]{//},
  morecomment=[n]{/*}{*/},
  morestring=[b]",
  morestring=[b]',
  morestring=[b]"""
}
\lstdefinelanguage{haskell}{
  morekeywords={class,instance,where,do,data,newtype,default,deriving,module},
  otherkeywords={<-},
  sensitive=true,
  morecomment=[l]{--},
  morecomment=[n]{\{-}{-\}}, 
  morestring=[b]",
  morestring=[b]',
  morestring=[b]"""
}
\lstdefinestyle{scala}{
  language=scala,
  basicstyle=\footnotesize\ttfamily,
  stringstyle=\color{darkgreen}\ttfamily,
  commentstyle=\color{gray}\ttfamily,
  keywordstyle=\footnotesize\color{blue}\ttfamily,
  tabsize=2,
  moredelim=**[is][\btHL]{`}{`}
}
\lstdefinestyle{haskell}{
  language=haskell,
  basicstyle=\tiny\ttfamily,
  stringstyle=\color{darkgreen}\ttfamily,
  commentstyle=\color{gray}\ttfamily,
  keywordstyle=\tiny\color{blue}\ttfamily,
  tabsize=2
}
\lstdefinelanguage{JavaScript}{
  keywords={typeof, new, true, false, catch, function, return, null, catch, switch, var, if, in, while, do, else, case, break},
  keywordstyle=\color{blue}\bfseries,
  ndkeywords={class, export, boolean, throw, implements, import, this},
  ndkeywordstyle=\color{darkgray}\bfseries,
  identifierstyle=\color{black},
  sensitive=false,
  comment=[l]{//},
  morecomment=[s]{/*}{*/},
  commentstyle=\color{purple}\ttfamily,
  stringstyle=\color{red}\ttfamily,
  morestring=[b]',
  morestring=[b]"
}

\lstdefinestyle{javascript}{
   language=JavaScript,
   extendedchars=true,
   basicstyle=\footnotesize\ttfamily,
   showstringspaces=false,
   showspaces=false,
   tabsize=4,
   breaklines=true,
   showtabs=false,
   captionpos=b
}
\maketitle

\begin{frame}{Agenda}
  \begin{itemize}
  \item Ground Rules
  \item Constraints Already Imposed By FP
  \item Why They Aren't Enough?
  \item What Needs To Be Done?
  \item Conclusion
  \item Questions?
  \end{itemize}
\end{frame}

\begin{frame}{Ground Rules}
  Properties of programs considered
  \begin{itemize}
  \item Reasoning
  \item Readability
  \item Reusability
  \end{itemize}
  Properties of programs not considered
  \begin{itemize}
  \item Performance
  \end{itemize}
\end{frame}

\begin{frame}{Constraints Already Imposed By FP}
  \begin{itemize}
  \item Immutability
  \item Referential Transparency (Unmodifiable Parameters to Functions)
  \end{itemize}
\end{frame}

\begin{frame}{Why Are They Necessary?}
  \begin{itemize}
  \item Equational Reasoning
  \item Ease of Refactoring
  \end{itemize}
\end{frame}

\begin{section}{Why Aren't They Enough?}

\begin{frame}[fragile]{Let's take a simple function}
  Can you guess what it does just by looking at it?
  \begin{lstlisting}[style=javascript]
    function add(var a, var b) {
      //...
    }
  \end{lstlisting}
\end{frame}

\begin{frame}{How did we do it?}
  \begin{center}
    \begin{itemize}
    \item Name Inference
    \item Number of Arguments
    \item Past Experience With Similar Functions
    \item ???
    \end{itemize}
\end{center}
\end{frame}

\begin{frame}[fragile]{But wait...}
    \begin{lstlisting}[style=javascript]
    function add(var a, var b) {
      return a.toString() + b.toString();
    }
  \end{lstlisting}
\end{frame}

\begin{frame}{What Happened There?}
  \begin{itemize}
    \item No real way to understand code without looking under the covers
    \item Every other form of reasoning can and will fail at some point
    \item There simply isn't enough information for the reader
    \item Gives the implementor too much power
  \end{itemize}
\end{frame}

\end{section}

\begin{section}{What Needs To Be Done?}
\begin{frame}[fragile]{Types to the Rescue}
  \begin{lstlisting}[style=scala]
    def add(a: Int, b: Int): Int = {
      //...
    }
  \end{lstlisting}
\end{frame}

\begin{frame}[fragile]{Much Better}
  \begin{itemize}
  \item Function Returns an Int
  \item So it is safe to assume it does what it says? 
  \end{itemize}
\end{frame}

\begin{frame}[fragile]{But Wait...}
  \begin{lstlisting}[style=scala]
    def add(a: Int, b: Int): Int = {
      a - b
    }
  \end{lstlisting}
\end{frame}

\begin{frame}[fragile]{Types Alone Aren't Enough}
  \begin{itemize}
  \item Gives implementor too much knowledge
  \item Gives user/reader too little info about implementation
  \end{itemize}
\end{frame}

\end{section}

\begin{section}{Parametricity}
\begin{frame}
\frametitle{The Journey}
\framesubtitle{Theorems for Free!}
\begin{block}{Philip Wadler \cite{wadler1989theorems} tells us:}
\begin{quotation}
Write down the definition of a polymorphic function on a piece of paper. Tell me its type, but be careful not to let me see the function's definition. I will tell you a theorem that the function satisfies.

The purpose of this paper is to explain the trick.
\end{quotation}
\end{block}
\end{frame}
\end{section}

\begin{frame}
\frametitle{References}

\bibliographystyle{amsalpha}
\bibliography{parametricity}

\end{frame}

\begin{frame}
Thank you!
\end{frame}

\end{document}
