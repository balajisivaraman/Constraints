\documentclass[bigger]{beamer}
\usepackage[utf8]{inputenc}
\usepackage{fixltx2e}
\usepackage{graphicx}
\usepackage{grffile}
\usepackage{longtable}
\usepackage{wrapfig}
\usepackage{rotating}
\usepackage[normalem]{ulem}
\usepackage{amsmath}
\usepackage{textcomp}
\usepackage{amssymb}
\usepackage{capt-of}
\usepackage{hyperref}
\usepackage{biblatex}
\usepackage{minted}
\usepackage{inconsolata}
\setminted{autogobble=true,fontsize=\footnotesize}
\addbibresource{constraints.bib}

\setbeamersize{text margin left=0.6cm,text margin right=0.6cm}

\usetheme{Copenhagen}

\author{Balaji Sivaraman (\href{https://twitter.com/balajisivaraman}{@balajisivaraman})}
\date{November 3, 2016}
\title{Free Yourself With Constraints}
\institute{ThoughtWorks}
\hypersetup{
 pdfauthor={Balaji Sivaraman},
 pdftitle={Free Yourself With Constraints},
 pdfkeywords={functional programming, parametricity, typed fp, higher-kinded types},
 pdfsubject={A talk on how constraints help us write readable programs},
 pdflang={English}}

\begin{document}

\maketitle

\begin{frame}{About Me}
  \begin{itemize}
  \item Primarily worked on Java/Spring stack
  \item Bitten by the FP bug in 2012 thanks to Scala
  \item Currently have an affinity for Functional Programming/Type Theory
  \item Occasionally dabble with Haskell, Idris, Purescript etc.
  \end{itemize}
\end{frame}

\begin{section}{Introduction}
\begin{frame}{Agenda}
  \begin{itemize}
  \item Introduction \cite{TMorrisParametricity} \cite{KNuttycombeParametricity}
  \item Ground Rules
  \item Constraints Already Imposed By FP
  \item Why They Aren't Enough?
  \item What Needs To Be Done?
  \item Conclusion
  \item Questions?
  \end{itemize}
\end{frame}

\begin{frame}{Ground Rules}
  Properties of programs considered
  \begin{itemize}
  \item Reasoning
  \item Readability
  \item Reusability
  \end{itemize}
  Properties of programs not considered
  \begin{itemize}
  \item Performance
  \end{itemize}
\end{frame}
\end{section}

\begin{section}{Constraints Already Imposed By FP}

\begin{frame}{Constraints Already Imposed By FP}
  \begin{itemize}
  \item Pure Functions - Code should be series of function calls instead of instruction executions
  \item Immutability - Functions cannot modify global variables, throw errors etc.
  \end{itemize}
\end{frame}

\begin{frame}{Why Are They Necessary?}
  \begin{itemize}
  \item Referential Transparency - Functions must produce same output given same input values
  \item Equational Reasoning - Function calls can be replaced by the values they compute to understand code more easily
  \item Ease of Refactoring
  \end{itemize}
\end{frame}

\end{section}

\begin{section}{Why Aren't They Enough?}

\begin{frame}[fragile]{Let's take a simple function}
  Can you guess what this function does just by looking at it?
  \begin{minted}{javascript}
    function add(var a, var b) {
      //...
    }
  \end{minted}
\end{frame}

\begin{frame}{How did we do it?}
  \begin{center}
    \begin{itemize}
    \item Name Inference
    \item Number of Arguments
    \item General Experience and Common Sense
    \end{itemize}
\end{center}
\end{frame}

\begin{frame}[fragile]{But wait...}
  \begin{minted}{javascript}
    function add(var a, var b) {
      return a.toString() + b.toString();
    }
  \end{minted}
\end{frame}

\begin{frame}{What Happened There?}
  \begin{itemize}
    \item Necessity to look under the covers
    \item Other forms of reasoning can and will fail at some point
    \item Not enough information for the reader
    \item Too much power to the implementor
    \item Mentally constrained input values
  \end{itemize}
\end{frame}

\end{section}

\begin{section}{What Needs To Be Done?}
\begin{frame}[fragile]{Types to the Rescue}
  \begin{minted}{scala}
    def add(a: Int, b: Int): Int = {
      //...
    }
  \end{minted}
\end{frame}

\begin{frame}[fragile]{Much Better}
  \begin{itemize}
  \item Function takes 2 Integers
  \item And returns an Integer
  \item Mental constraints have been made explicit
  \item So it is safe to assume it does what it says?
  \end{itemize}
\end{frame}

\begin{frame}[fragile]{But Wait...}
  \begin{minted}{scala}
    def add(a: Int, b: Int): Int = {
      a - b
    }
  \end{minted}
\end{frame}

\begin{frame}[fragile]{Types Alone Aren't Enough}
  \begin{itemize}
  \item Still too much power to the implementor
  \item Still not enough information for the reader
  \end{itemize}
\end{frame}

\end{section}

\begin{section}{Parametricity}
\begin{frame}
\frametitle{Parametricity}
\begin{block}{Philip Wadler \cite{Wadler89theoremsfor} writes:}
\begin{quotation}
Write down the definition of a polymorphic function on a piece of paper. Tell me its type, but be careful not to let me see the function's definition. I will tell you a theorem that the function satisfies.

The purpose of this paper is to explain the trick.
\end{quotation}
\end{block}
\end{frame}

\begin{frame}[fragile]
\frametitle{A Simple Parametrically Polymorphic Function}
  Can you guess what this function does just by looking at it?
  \begin{minted}{scala}
    def doSomething[A](a: A, b: A): A = {
      //...
    }
  \end{minted}
\end{frame}

\begin{frame}{How did we do it?}
  \begin{center}
    \begin{itemize}
    \item Power of parametric polymorphism
    \item Very little power for the implementor
    \item Lot of info available to the reader from definition
    \end{itemize}
\end{center}
\end{frame}

\begin{frame}[fragile]
\frametitle{Another (familiar?) Parametrically Polymorphic Function}
  Can you guess what this function does just by looking at it?
  \begin{minted}{scala}
    def doSomethingElse[A, B](a: List[A])(f: A => B): List[B] =
    {
      //...
    }
  \end{minted}
\end{frame}

\begin{frame}[fragile]{But wait...}
  \begin{minted}{scala}
    def doSomethingElse[A, B](a: List[A])(f: A => B): List[B] =
    {
      List(f(a.head))
    }
  \end{minted}
\end{frame}

\begin{frame}{What Happened There?}
  \begin{itemize}
    \item Parametric polymorphism alone isn't enough
    \item Implementor sometimes still has too much info to work with
    \item Reader sometimes still has to look under the covers to feel really safe
  \end{itemize}
\end{frame}

\end{section}

\begin{section}{Type Classes and Higher-Kinded Types}
\begin{frame}[fragile]
  \frametitle{Functor Typeclass}
  \begin{minted}{scala}
    trait Functor[F[_]] {
      def map[A, B](fa: F[A])(f: A => B): F[B]
    }
  \end{minted}
  \begin{itemize}
  \item{Here \mintinline{scala}{F[_]} denotes a higher-kinded type}
  \item Think of them as analogous to higher-order functions at the type level, i.e. they take a type themselves
  \end{itemize}
\end{frame}

\begin{frame}[fragile]
  \frametitle{Why Type Classes and HKTs?}
  \begin{itemize}
  \item Instances are created for the Functor typeclass for each \mintinline{scala}{F[_]}, i.e. List, Option etc.
  \item These instances are distinct entities, completely separate from the Typeclass itself
  \item Just like HOFs, HKTs give you the ability to abstract over the type constructor itself, which has numerous benefits in practice
  \end{itemize}
\end{frame}

\begin{frame}[fragile]
  \frametitle{Different Instances}
  \begin{minted}{scala}
    implicit object ListFunctor extends Functor[List] {
      def map[A, B](fa: List[A])(f: A => B): List[B] = fa.map(f)
    }

    implicit object OptionFunctor extends Functor[Option] {
      def map[A, B](fa: Option[A])(f: A => B): Option[B] = fa.map(f)
    }
  \end{minted}
\end{frame}

\begin{frame}[fragile]
  \frametitle{How is this better?}
  \begin{itemize}
  \item Only one instance of each class in our entire application
  \item Typeclasses are a purely compile-time construct, search and verification of instances is at compile-time
  \item Functor laws as property tests (QuickCheck, ScalaCheck etc.)
  \item Created instances have to adhere to Functor laws
  \end{itemize}
\end{frame}

\begin{frame}[fragile]
  \frametitle{Reusability}
  \begin{minted}{scala}
    trait Functor[F[_]] {
      def map[A, B](fa: F[A])(f: A => B): F[B]

      def lift[A, B](f: A => B): F[A] => F[B] = ???
    }
  \end{minted}
\end{frame}

\begin{frame}[fragile]
  \frametitle{Reusability}
  \begin{minted}{scala}
    trait Functor[F[_]] {
      def map[A, B](fa: F[A])(f: A => B): F[B]

      def lift[A, B](f: A => B): F[A] => F[B] = fa => map(fa)(f)
    }
  \end{minted}
\end{frame}

\begin{frame}[fragile]
  \frametitle{Reusability}
  \centerline{
    \huge{Code}
  }
\end{frame}
\end{section}

\begin{frame}[fragile]
  \frametitle{Reusability}
  \begin{itemize}
  \item Many typeclasses already available (Scalaz, Cats) - Functor, Applicative, Monad, Traverse etc
  \item Easy to define instances for our classes or new typeclasses themselves
  \item Typeclasses, by definition, are parametrically polymorphic giving them many compile time benefits
  \end{itemize}
\end{frame}

\begin{section}{Conclusion}
\begin{frame}
\footnotesize
\frametitle{Conclusion}
\begin{itemize}
\item Code we write is always constrained, whether implicitly or explicitly
\item Making them explicit leads to better ways to reason about and understand our code
\item The more constrained our code, the more easy to read and refactor it is
\item Parametricity is the bare minimum constraint we should work with
\item Prefer typeclasses over inheritance if the language allows it
\item Higher kinded types enable a lot of reuse
\item Always reach for the most constrained abstraction, leaving the implementor with fewer choices
\end{itemize}
\end{frame}
\end{section}


\begin{frame}
\frametitle{References}

\printbibliography

\end{frame}

\begin{frame}
  \centerline{
    \huge{Thank you!}
  }
  \centerline{
    \footnotesize{Slides source available at: \href{https://github.com/balajisivaraman/constraints}{https://github.com/balajisivaraman/constraints}}
  }
\end{frame}
\end{document}
